%Настройка шрифта
\newgeometry{a4paper,top=1cm,bottom=2cm,left=2cm,right=2cm,marginparwidth=1.25cm}
\justifying
\small
\setstretch{0.4}
%Наcтройка нумерации страниц
\fancyhf{} % clear all header and footers
\renewcommand{\headrulewidth}{0pt} % remove the header rule
\lfoot{\thepage}
\pagestyle{fancy}
\setcounter{page}{18}
%Особая колонка для поддержки центрирования одновременно с заданием ширины столбца
\newcolumntype{C}[1]{>{\centering\let\newline\\\arraybackslash\hspace{0pt}}m{#1}}
%-----------Настройки для paracol--------------
\columnratio{0.3}
\setlength{\columnsep}{20pt}
\tolerance=1
\emergencystretch=\maxdimen
\hyphenpenalty=10000
\hbadness=10000
%----------------------------------------------
\begin{paracol}{2}
{\it
\parб) Пусть $K_0$ состоит из трех вершин правильного треугольника площади 1. Найдите площадь наименьшего выпуклого многоугольника, содержащего множество $K_n (n=1,2,...)$.
\parВ следующих трех пунктах $K_0$ - множество четырех вершин правильного тетраэдра объема 1.
\parв)Рассмотрим наименьший выпуклый многогранник, содержащий все точки множества $K_1$. Сколько и каких граней у этого многогранника?
\parг)Чему равен объем этого многогранника?
\parд)Найдите объем наименьшего ввыпуклого многогранника, содержащего множество $K_n (n=2,3,...)$.}
%-------------------------------------------------
\begin{figure}[h]

\centering
\setlength{\abovecaptionskip}{0pt}
\setlength{\belowcaptionskip}{0pt}

\includegraphics[width=1.0\linewidth, height=0.55\linewidth]{images/img1.png}
\caption{\label{fig:ris1}}

\includegraphics[width=1.0\linewidth, height=0.55\linewidth]{images/img2.png}
\caption{\label{fig:ris2}}

\end{figure}
%-------------------------------------------------
\begin{table}[h]

\centering
\setlength{\abovecaptionskip}{0pt}
\setlength{\belowcaptionskip}{0pt}

\begin{tabular}{C{1.1cm} | C{1.1cm}}
$n$ & $3^{n}$ \\
\hline
1 & 3 \\
2 & 9 \\
3 & 27 \\
4 & 81 \\
5 & 243 \\
6 & 729 \\
7 & 2187 \\
8 & 6561 \\
9 & 19683 \\
\end{tabular}

\caption{\label{tab:table1}}
\end{table}
%-------------------------------------------------
\begin{figure}[h]

\centering
\setlength{\abovecaptionskip}{0pt}
\setlength{\belowcaptionskip}{0pt}

\includegraphics[width=1.0\linewidth, height=0.55\linewidth]{images/img3.png}
\caption{\label{fig:ris3}}

\end{figure}
%-------------------------------------------------
\switchcolumn
\parПрежде чем читать решение дальше, нарисуйте фигуры $W(\Phi)$ и $W^{2}(\Phi)$, если $\Phi$ - пара точек; отрезок; прямоугольник; четыре вершины прямоугольника; три вершины треугольника. А каково $W(\Phi)$, если $\Phi$ - куб? восемь вершин куба?
\parНайчем с <<одномерного>> случая, когда фигура $\Phi$ лежит на прямой. Этот случай нужен для решения задачи а), но, как мы увидим, он пригодиться и в более трудных задачах.
\parПусть наша фигура - это отрезок $\Pi=[AB]$ длинны $d$ с серединой в точке $O$. Тогда $\Phi(\Pi)=[A_{1}B_{1}]$ - отрезок длинны $3d$ с серединой в той же точке $O$. Отсюда сразу следует, что $\Phi^{2}(\Pi)={A_{2}B_{2}}$, где $|A_{2}B_{2}|=9d$, и вообще $\Phi^{n}(\Pi)={A_{n}B_{n}}$, где $|A_{n}B_{n}|=3^{n}d$, причем все отрезки $[A_{n}B_{n}]$ имеют середину в той же точке $O$ (рис. \ref{fig:ris1}).
\parБудем в дальнейшем образ фигуры $\Phi$, имеющей центр симметрии $O$, при гомотетии с коэффикиентом $k$ и центром $O$ обозначать через $k\Phi$. Мы выяснили, сто если $\Pi$ - отрезок, то $W(\Pi)=3\Pi$ и вообще
\begin{equation}
    W^{n}(\Pi)=3^{n}\Pi.\tag{3}\label{eq:3}
\end{equation}
Забегая вперед, заметим, что это верно для любой \textit{центрированно-симмметричной выпуклой} фигуры $\Pi$. Но прежде закончим с одномерным случаем - ответим на вопрос а).
\parа) Пусть $K_{0}=(A,B),|AB|=1$. Согласно (2), где $\Phi=(A,B), \Pi=\Phi'=[AB]$,
\begin{equation}
    K_{n}=\Phi^{n}(K_{0})\subset\Phi^{n}(\Pi)=3^{n}\Pi.\tag{4}\label{eq:4}
\end{equation}
При этом ясно, что $K_{n}$ содержит все точки отрезка $[A_{n}B_{n}]=3^{n}\Pi$, находящиеся на целочисленном расстоянии от точки $A$, причем точки ${A_{n},B_{n}}=3^{n}{A,B}$ находятся от $A$ на расстоянии $(3^{n}-1)/2$ и $(3^{n}+1)/2$ (рис. \ref{fig:ris2}). Поскольку эти числа (Таблица \ref{tab:table1}) больше 10 000 при $n\geqslant10$, ответ на вопрос а): $n=10$.
\parВ решении плоских и пространственных задач про операцию $W$ очень полезно такое свойство: \textit{если $\Phi_{l}$ - проекция фигуры $\Phi$ на прямую l, то}
\begin{equation}
    W(\Phi_{l})=(W(\Phi))_{l}\tag{5}\label{eq:5}
\end{equation}
(проекция $W(\Phi)$ на прямую $l$ совпадает с результатом применения $W$ к проекции $\Phi_{l}$) 
{\renewcommand{\thefootnote}{*)}\footnote{
Проекцией пространственной фигуры $\Phi$ на прямую $l$ называется множество точек $M \in l$ таких, что плоскость, проходящая через точку $M$ и перпендикулярная к $l$, пересекается с $\Phi$.
}}
Это сразу следует из оперделения $W$ и того факта, что при проекции точки, симметричные относительно точки $M$, попадают и точки, симметричный относительно ее проекции $M_{l}$ (рис. \ref{fig:ris3}).
\parОтсюда вытекает, что если $\Pi$ - полоса, а $3^{n}\Pi$ - ее образ при гомотетии с коэффициентом $3^{n}$ относительно центра симметрии (какой-либо \textit{точки на средней прямой}), то выполнено соотношение (\ref{eq:3}). Это верно и для <<пространственной полосы>> $\Pi$ - множества точек, заключенных между плоскостями.
\parИз (2), (\ref{eq:3}), (\ref{eq:5}) и решения задачи а) следует также, что если мы заключаем фигуру $\Phi$ в \textit{опорную полосу} $\Pi$ какого-либо направления - такую полосу $\Pi \supset \Phi$, края которой содежат все точки их $\Phi$, - то \textit{полоса $3^{n}\Pi \supset W^{n}(\Phi)$ будет опорной для $W^{n}(\Phi)$} (рис. 5). Это соображение, обобщающее (\ref{eq:4}), позволяет найти выпуклую оболочку $W(\Phi)$ для любой фигуры $\Phi$.
\parРазберем конкретные примеры.
\parб) Пусть $K_{0}={A,B,C}$ - вершины правильного треугольника, $K_{n}=W^{n}(K_{0}),n=1,2,...$(рис. 6). Докажем, что для любого $n\geqslant1$ выпуклая обочка множества $K_{n}$ получается так: нужно взять концы отрезков $3^{n}[AB],3^{n}[AC]$ и построить выпуклый шестиугольник Ш$_{n}$ с вершинами в этих точках.
\end{paracol}
