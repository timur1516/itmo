\documentclass[a4paper, 9pt]{article}
\usepackage[english, russian]{babel}
\usepackage[a4paper,top=2cm,bottom=2cm,left=2cm,right=2cm,marginparwidth=1.75cm]{geometry}
\usepackage[dvipsnames, svgnames, x11names]{xcolor}
\usepackage{tikz}
\usepackage{amssymb}
\usepackage{paracol}
\usepackage{tkz-euclide}
\usetikzlibrary{angles}
\pagestyle{empty} %  выключаенм нумерацию
\renewcommand{\baselinestretch}{1.5} 

\usepackage[T1,T2A]{fontenc}
\usepackage[utf8]{inputenc}
\usepackage[english,russian]{babel}
\usepackage{tempora}
\usepackage{indentfirst}

\definecolor{myOrange}{RGB}{217,77,26}
\definecolor{myBlue}{RGB}{38,89,153}
\definecolor{myYellow}{RGB}{242,179,26}

%Команда для рисовки линий
\newcommand\MyLine[4]{
\begin{tikzpicture}[line width=#1]
\draw[#2]
(0,0) node[font=\scriptsize, text=black, above] {#3} -- 
(1.5,0) node[font=\scriptsize, text=black, above] {#4};
\end{tikzpicture}
}
%Команада для углов 1
\newcommand{\MyAngleOne}[5]{
\begin{tikzpicture}[baseline=(current bounding box.center)]
\fill[#1] (0.7,0) node[font=\scriptsize, text=black, right] {#4} -- 
(0,0) node[font=\scriptsize, text=black, left] {#3} 
arc (180:#2:0.7) node[font=\scriptsize, text=black, above] {#5} -- cycle;
\end{tikzpicture}
}
%Команада для углов 2
\newcommand{\MyAngleTwo}[5]{
\begin{tikzpicture}[baseline=(current bounding box.center)]
\fill[#1] 
(0,0) node[font=\scriptsize, text=black, left] {#3} -- 
(0.7,0) node[font=\scriptsize, text=black, right] {#4} 
arc (0:#2:0.7) node[font=\scriptsize, text=black, above] {#5} -- cycle;
\end{tikzpicture}
}

\begin{document}

\columnratio{0.75}
\setlength{\columnsep}{5pt}
\tolerance=1
\emergencystretch=\maxdimen
\hyphenpenalty=10000
\hbadness=10000

\begin{paracol}{2}
\begin{center}
\hfill \LargeКНИГА I ПРЕДЛ. XXVI. ТЕОРЕМА \hfill{5\raisebox{-0.2em}{1}}\\
\vspace{1cm}
\LargeСлучай II.\\
Теперь пусть \MyLine{3pt}{myOrange}{C}{A} = \MyLine{1.5pt}{myOrange}{F}{D}, лежат\\
против равных углов
\MyAngleOne{myOrange}{120}{A}{B}{C}
и
\MyAngleOne{myOrange}{120}{D}{E}{F}
.\\
Если такое возможно, пусть\\
\begin{tikzpicture}[line width=1.5pt]
\coordinate (D) at (0,0);
\coordinate (G) at (1,0);
\coordinate (E) at (1.5,0);
\draw[myBlue] (D) -- (G);
\draw[myBlue, dashed] (G) -- (E);
\tkzLabelPoint[above](D){\scriptsize{D}}
\tkzLabelPoint[above](E){\scriptsize{E}}
\end{tikzpicture}
> 
\MyLine{3pt}{myBlue}{A}{B}
, тогда возьмём\\
\MyLine{1.5pt}{myBlue}{D}{G}
=
\MyLine{3pt}{myBlue}{A}{B}
, проведём 
\MyLine{1.5pt}{myYellow}{F}{G}
.\\
Тогда в
\begin{tikzpicture}[line width=3pt] 
\coordinate (A) at (0,0);
\coordinate (B) at (1.5,0);
\coordinate (C) at (0.5,1.5);
\draw[myBlue] (A) -- (B);
\draw[Black] (B) -- (C);
\draw[myOrange] (C) -- (A);

\tkzLabelPoint[left](A){\scriptsize{A}}
\tkzLabelPoint[right](B){\scriptsize{B}}
\tkzLabelPoint[above](C){\scriptsize{C}}

\end{tikzpicture}
и
\begin{tikzpicture}[line width=1.5pt] 
\coordinate (D) at (0,0);
\coordinate (G) at (1,0);
\coordinate (F) at (0.5,1.5);
\draw[myBlue] (D) -- (G);
\draw[myYellow] (G) -- (F);
\draw[myOrange] (F) -- (D);

\tkzLabelPoint[left](D){\scriptsize{D}}
\tkzLabelPoint[right](G){\scriptsize{G}}
\tkzLabelPoint[above](F){\scriptsize{F}}

\end{tikzpicture}\\
получим 
\MyLine{3pt}{myOrange}{C}{A} 
= 
\MyLine{1.5pt}{myOrange}{F}{D},\\
\MyLine{3pt}{myBlue}{A}{B} 
= 
\MyLine{1.5pt}{myBlue}{D}{G} 
и
\MyAngleTwo{myYellow}{75}{A}{B}{C}
=
\MyAngleTwo{myYellow}{75}{D}{E}{F},\\
$\therefore$
\MyAngleOne{myOrange}{120}{A}{B}{C}
=
\MyAngleOne{Black}{105}{D}{G}{F}
(пр. I.$_{4}$)\\
но
\MyAngleOne{myOrange}{120}{A}{B}{C}
=
\MyAngleOne{myOrange}{120}{D}{E}{F}
(гип.)\\
$\therefore$
\MyAngleOne{Black}{105}{D}{G}{F}
=
\MyAngleOne{myOrange}{120}{D}{E}{F}
что не имеет смысла (пр. I.$_{16}$)\\
Следовательно, ни \MyLine{3pt}{myBlue}{A}{B} ни
\begin{tikzpicture}[line width=1.5pt]
\coordinate (D) at (0,0);
\coordinate (G) at (1,0);
\coordinate (E) at (1.5,0);
\draw[myBlue] (D) -- (G);
\draw[myBlue, dashed] (G) -- (E);

\tkzLabelPoint[above](D){\scriptsize{D}}
\tkzLabelPoint[above](E){\scriptsize{E}}
\end{tikzpicture}
не больше другой, а значит они равны. Следовательно (согласно пр.I.$_{4}$) треугольники равны во всех отношениях.
\end{center}
\begin{flushright}
ч.т.д.
\end{flushright}

\switchcolumn

\vspace*{1cm}
\begin{center}
\begin{tikzpicture}[line width=3pt] 
\coordinate (A) at (0,0);
\coordinate (B) at (4.5,0);
\coordinate (C) at (1.5,4.5);
\path
pic[fill=myOrange,angle radius=9mm] {angle = C--B--A}
pic[fill=myYellow,angle radius=9mm] {angle = B--A--C};
\draw[myBlue] (A) -- (B);
\draw[Black] (B) -- (C);
\draw[myOrange] (C) -- (A);

\tkzLabelPoint[left](A){\scriptsize{A}}
\tkzLabelPoint[right](B){\scriptsize{B}}
\tkzLabelPoint[above](C){\scriptsize{C}}

\end{tikzpicture}
\hfill \break
\begin{tikzpicture}[line width=1.5pt] 
\coordinate (D) at (0,0);
\coordinate (E) at (4.5,0);
\coordinate (F) at (1.5,4.5);
\coordinate (G) at (3,0);
\path
pic[fill=myOrange,angle radius=9mm] {angle = F--E--D}
pic[fill=myYellow,angle radius=9mm] {angle = E--D--F}
pic[fill=black,angle radius=9mm] {angle = F--G--D};
\draw[myBlue] (D) -- (G);
\draw[myBlue, dashed] (G) -- (E);
\draw[Black] (E) -- (F);
\draw[myOrange] (F) -- (D);
\draw[myYellow] (F) -- (G);

\tkzLabelPoint[left](D){\scriptsize{D}}
\tkzLabelPoint[right](E){\scriptsize{E}}
\tkzLabelPoint[above](F){\scriptsize{F}}
\tkzLabelPoint[below](G){\scriptsize{G}}

\end{tikzpicture}

\end{center}
\end{paracol}
\end{document}